

\documentclass[11pt,a4paper,sans]{moderncv}        % possible options include font size ('10pt', '11pt' and '12pt'), paper size ('a4paper', 'letterpaper', 'a5paper', 'legalpaper', 'executivepaper' and 'landscape') and font family ('sans' and 'roman')

% moderncv themes
\moderncvstyle{casual}                             % style options are 'casual' (default), 'classic', 'banking', 'oldstyle' and 'fancy'
\moderncvcolor{green}                               % color options 'black', 'blue' (default), 'burgundy', 'green', 'grey', 'orange', 'purple' and 'red'
%\renewcommand{\familydefault}{\sfdefault}         % to set the default font; use '\sfdefault' for the default sans serif font, '\rmdefault' for the default roman one, or any tex font name
%\nopagenumbers{}                                  % uncomment to suppress automatic page numbering for CVs longer than one page

% character encoding
%\usepackage[utf8]{inputenc}                       % if you are not using xelatex ou lualatex, replace by the encoding you are using
%\usepackage{CJKutf8}                              % if you need to use CJK to typeset your resume in Chinese, Japanese or Korean

% adjust the page margins
\usepackage[scale=0.75]{geometry}
%\setlength{\hintscolumnwidth}{3cm}                % if you want to change the width of the column with the dates
%\setlength{\makecvtitlenamewidth}{10cm}           % for the 'classic' style, if you want to force the width allocated to your name and avoid line breaks. be careful though, the length is normally calculated to avoid any overlap with your personal info; use this at your own typographical risks...

% personal data
\name{Julio César}{Sosa Mondragón}
%\title{CV}                               % optional, remove / comment the line if not wanted
\address{Manuel M. López}{CP 13300}{Del. Tláhuac, Cd. de México.}% optional, remove / comment the line if not wanted; the "postcode city" and "country" arguments can be omitted or provided empty
\phone[mobile]{(712)-186-13-72}                   % optional, remove / comment the line if not wanted; the optional "type" of the phone can be "mobile" (default), "fixed" or "fax"
\phone[fixed]{(55)-5850-3954}
%\phone[fax]{+3~(456)~789~012}
\email{julio\_somo@ciencias.unam.mx}                               % optional, remove / comment the %line if not wanted
%\homepage{www.johndoe.com}                         % optional, remove / comment the line if not wanted
\social[linkedin]{juliosomo}                        % optional, remove / comment the line if not wanted
%\social[twitter]{jdoe}                             % optional, remove / comment the line if not wanted
\social[github]{gmtw}                              % optional, remove / comment the line if not wanted
%\extrainfo{additional information}                 % optional, remove / comment the line if not wanted
\photo[64pt][0.4pt]{color}                       % optional, remove / comment the line if not wanted; '64pt' is the height the picture must be resized to, 0.4pt is the thickness of the frame around it (put it to 0pt for no frame) and 'picture' is the name of the picture file
%\quote{Some quote}                                 % optional, remove / comment the line if not wanted

% bibliography adjustements (only useful if you make citations in your resume, or print a list of publications using BibTeX)
%   to show numerical labels in the bibliography (default is to show no labels)
\makeatletter\renewcommand*{\bibliographyitemlabel}{\@biblabel{\arabic{enumiv}}}\makeatother
%   to redefine the bibliography heading string ("Publications")
%\renewcommand{\refname}{Articles}

% bibliography with mutiple entries
%\usepackage{multibib}
%\newcites{book,misc}{{Books},{Others}}
%----------------------------------------------------------------------------------
%            content
%----------------------------------------------------------------------------------
\begin{document}
%\begin{CJK*}{UTF8}{gbsn}                          % to typeset your resume in Chinese using CJK
%-----       resume       ---------------------------------------------------------
\makecvtitle

\section{Educación}
\cventry{2012-2018}{Físico}{UNAM}{pasante}{}{}  % arguments 3 to 6 can be left empty
%\cventry{2009--2012}{Preparatoria}{Escuela Preparatoria Oficial 109}{}{}{}
\cvitemwithcomment{Inglés}{B2}{}

%\section{Master thesis}
%\cvitem{title}{\emph{Title}}
%\cvitem{supervisors}{Supervisors}
%\cvitem{description}{Short thesis abstract}

\section{Experiencia}
\subsection{Institución Liceo}
\cventry{2015--2016}{Actividades}{}{}{}{Generador y evaluador de examenes para niveles de preparatoria y secundaria}
%Detailed achievements:%
%\begin{itemize}%
%\item Achievement 1;
%\item Achievement 2, with sub-achievements:
%  \begin{itemize}%
%  \item Sub-achievement (a);
%  \item Sub-achievement (b), with sub-sub-achievements (don't do this!);
%    \begin{itemize}
%    \item Sub-sub-achievement i;
%   \item Sub-sub-achievement ii;
%    \item Sub-sub-achievement iii;
%    \end{itemize}
%  \item Sub-achievement (c);
%  \end{itemize}
%\item Achievement 3.
%\end{itemize}}
%\cventry{year--year}{Job title}{Employer}{City}{}{Description line 1\newline{}Description line 2}
%\subsection{Miscellaneous}
%\cventry{year--year}{Job title}{Employer}{City}{}{Description}

%\section{Idiomas}
%\cvitemwithcomment{Inglés}{B2}{}
%\cvitemwithcomment{Español}{Nativo}{}
%\cvitemwithcomment{Language 3}{Skill level}{Comment}

\section{Computación}
\cvdoubleitem{Python}{2 años de experiencia}{80 \%}{}
%\cvdoubleitem{PostgreSQL}{1 año de experiencia}{}{}
\cvdoubleitem{Fortran}{2 años de experiencia}{40 \%}{}
\cvdoubleitem{Git}{1 año de experiencia}{50 \%}{}
\cvdoubleitem{Linux/Ubuntu}{2 años de experiencia}{70 \%}{}
\cvdoubleitem{Latex}{2 años de experiencia}{70 \%}{}
\cvdoubleitem{Hojas de Cálculo}{2 años de experiencia}{40 \%}{}
\section{Habilidades}
\cvitem{Trabajo en equipo}{Excelente relación con mis compañeros de trabajo para poder alcanzar las metas en tiempo y forma.}
\cvitem{Autodidacta}{Aprendí python y ciencia de datos por mi cuenta. Además de que siempre estoy en constante aprendizaje y abierto hacia las nuevas tecnologías.}
\cvitem{Constancia}{Compromiso para realizar mis objetivos y mejorarlos continuamente.}
\cvitem{Accesibilidad}{Apoyo a otras áreas.}
\section{Intereses}
\cvitem{Data science}{Me interesa la programación combinada con estadística o modelado matemático.}
\cvitem{Deep Learning}{Conocer las redes neuronales a fondo y aprender a aplicarlos a diferentes ramas de la salud.}
%\cvitem{hobby 3}{Description}
%\section{Extra 1}
%\cvlistitem{Item 1}
%\cvlistitem{Item 2}
%\cvlistitem{Item 3. This item is particularly long and therefore normally spans over several lines. Did you notice the indentation when the line wraps?}

%\section{Extra 2}
%\cvlistdoubleitem{Item 1}{Item 4}
%\cvlistdoubleitem{Item 2}{Item 5\cite{book1}}
%\cvlistdoubleitem{Item 3}{Item 6. Like item 3 in the single column list before, this item is particularly long to wrap over several lines.}
%
%\section{References}
%\begin{cvcolumns}
%  \cvcolumn{Category 1}{\begin{itemize}\item Person 1\item Person 2\item Person 3\end{itemize}}
%  \cvcolumn{Category 2}{Amongst others:\begin{itemize}\item Person 1, and\item Person 2\end{itemize}(more upon request)}
%  \cvcolumn[0.5]{All the rest \& some more}{\textit{That} person, and \textbf{those} also (all available upon request).}
%\end{cvcolumns}
%
%% Publications from a BibTeX file without multibib
%%  for numerical labels: \renewcommand{\bibliographyitemlabel}{\@biblabel{\arabic{enumiv}}}% CONSIDER MERGING WITH PREAMBLE PART
%%  to redefine the heading string ("Publications"): \renewcommand{\refname}{Articles}
%\nocite{*}
%\bibliographystyle{plain}
%\bibliography{publications}                        % 'publications' is the name of a BibTeX file

% Publications from a BibTeX file using the multibib package
%\section{Publications}
%\nocitebook{book1,book2}
%\bibliographystylebook{plain}
%\bibliographybook{publications}                   % 'publications' is the name of a BibTeX file
%\nocitemisc{misc1,misc2,misc3}
%\bibliographystylemisc{plain}
%\bibliographymisc{publications}                   % 'publications' is the name of a BibTeX file

\end{document}


%% end of file `template.tex'.
